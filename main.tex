\documentclass[a4paper]{article}
\usepackage[margin=0.5in]{geometry}
\usepackage{tikz}
\usepackage{rubikcube,rubikrotation,rubikpatterns}
\usepackage{multicol}
\usepackage{xstring}
\pagestyle{empty}
\overfullrule=5mm

\newcommand{\formatRubikSequence}[1]{
\StrSubstitute{#1}{,}{ }[\rubiktmp]%
\StrSubstitute{\rubiktmp}{p}{'}[\rubiktmp]%
\rubiktmp
}

% #1: drawings on the cube
% #2: name of the algorithm
% #3: sequence
\newcommand{\alg}[4][]{
	\RubikCubeSolved%
	\RubikRotation{x2,yp}%
	% sanitizing the sequence because the rubik package doesn’t like R2p
	\StrSubstitute{#3}{2p}{2}[\seqtmp]
	\StrSubstitute{\seqtmp}{r}{Rw}[\seqtmp]
	\StrSubstitute{\seqtmp}{u}{Uw}[\seqtmp]
	\StrSubstitute{\seqtmp}{f}{Fw}[\seqtmp]
	\StrSubstitute{\seqtmp}{b}{Bw}[\seqtmp]
	\StrSubstitute{\seqtmp}{l}{Lw}[\seqtmp]
	\StrSubstitute{\seqtmp}{d}{Dw}[\seqtmp]
	\RubikRotation{\seqtmp,<inverse>}%
	\noindent\ShowCube{2cm}{0.45}{%
		\DrawRubikFaceUpSide%
		#1
	}
	\hspace{0.2cm}\textbf{#2:}\formatRubikSequence{#3}
	\vspace{0.2cm}
}

\begin{document}
\begin{center}
	{\Huge{\textbf{2LLL (2 Look Last Layer)}}}
\end{center}

\section{OCLL (Orientation of Corners of Last Layer)}

	\algoll{Sune}{(R,U,Rp,U),R,U2p,Rp}{}

	\algoll{Antisune}{R,U2,Rp,Up,R,Up,Rp}{}

	\algoll{U}{R2p,Dp,(R,U2,Rp),D,(R,U2,R)}{}

	\algoll{T}{(r,U,Rp,Up),(rp,F,R,Fp)}{}

	\algoll{L}{(F,Rp,Fp),(r,U,R,Up),rp}{}

	\algoll{H}{(R,U,Rp,U),(R,Up,Rp,U),(R,U2p,Rp)}{}

	\algoll{Pi}{Rp,U2p,(R2,U,R2p,U),R2,U2p,Rp}{}

\section{CP (Corner Permutation)}

~

\alg[{
	\draw[<->,thick] (2.5,0.5) -- (2.5, 2.5);%
	\draw[<->,thick] (0.5,1.5) -- (2.4,1.5);%
}]{T perm}{R,U,Rp,Up,Rp,F,R2,Up,Rp,Up,R,U,Rp,Fp}{}

\alg[{
	\draw[<->,thick] (0.5,2.5) -- (2.5,0.5);%
	\draw[<->,thick] (0.5,1.5) -- (1.5,2.5);%
}]{Y perm}{F,R,Up,Rp,Up,R,U,Rp,Fp,R,U,Rp,Up,Rp,F,R,Fp}{}


\section{EPLL (Edge Permutation of the Last Layer)}

\begin{multicols}{2}
	\alg{Z perm}{Mp,Up,(M2,Up),(M2,Up),Mp,U2,M2}{}

	\alg{H perm}{M2,U,M2,U2,M2,U,M2}{}

	\alg[{
		\draw[<-,thick] (0.5,1.5) -- (2.3,1.5);%
		\draw[->,thick] (0.4,1.4) -- (1.5,0.5);%
		\draw[->,thick] (1.6,0.5) -- (2.5,1.4);%
	}]{Ua perm}{M2,U,M,U2,Mp,U,M2}{}

	\alg[{
		\draw[->,thick] (0.5,1.5) -- (2.3,1.5);%
		\draw[->,thick] (2.5,1.5) -- (1.6,0.5);%
		\draw[<-,thick] (0.5,1.4) -- (1.5,0.5);%
	}]{Ub perm}{M2,Up,M,U2,Mp,Up,M2}{}

\end{multicols}


\pagebreak
\section{PLL (Permutation of the Last Layer)}

\subsection{Adjacent corner swap}
\alg[{\draw[->, thick] (2.5, 0.5) -- (0.5,2.4);
	\draw[->, thick] (0.5,2.5) -- (2.5,2.5);
	\draw[->, thick] (2.5,2.4) -- (2.5,0.6);
}]{Aa perm}{x,Rp,U,Rp,D2,R,Up,Rp,D2,R2,xp}{}

\alg[{
	\draw[->,thick] (0.5,0.5) -- (2.5,0.5);
	\draw[->,thick] (2.5,0.6) -- (2.5,2.5);
	\draw[->,thick] (2.4,2.5) -- (0.5,0.6);
}]{Ab Perm}{x,R2,D2,R,U,Rp,D2,R,Up,R,xp}{}

\alg{F perm}{Rp,Up,Fp,R,U,Rp,Up,Rp,F,R2,Up,Rp,Up,R,U,Rp,U,R}{}

\alg{Ga perm}{R2,U,Rp,U,Rp,Up,R,Up,R2,Up,D,Rp,U,R,Dp}{}

\alg{Gb perm}{Rp,Up,R,U,Dp,R2,U,Rp,U,R,Up,R,Up,R2p,D}{}

\alg{Gc perm}{R2p,Up,R,Up,R,U,Rp,U,R2,U,Dp,R,Up,Rp,D}{}

\alg{Gd perm}{R,U,Rp,Up,D,R2,Up,R,Up,Rp,U,Rp,U,R2,Dp}{}

\alg{Ja perm}{x,R2p,F,R,Fp,R,U2p,rp,U,r,U2p,xp}{}

\alg{Jb perm}{R,U,Rp,Fp,R,U,Rp,Up,Rp,F,R2,Up,Rp}{}

\alg{Ra perm}{R,Up,Rp,Up,R,U,R,D,Rp,Up,R,Dp,Rp,U2,Rp}{}

\alg{Rb perm}{Rp,U2p,R,U2,Rp,F,R,U,Rp,Up,Rp,Fp,R2}{}

\alg{T perm}{R,U,Rp,Up,Rp,F,R2,Up,Rp,Up,R,U,Rp,Fp}{}


\subsection{Diagonal corner swap}
\alg{E perm}{xp,R,Up,Rp,D,R,U,Rp,Dp,R,U,Rp,D,R,Up,Rp,Dp,x}{}

\alg{Na perm}{R,U,Rp,U,R,U,Rp,Fp,R,U,Rp,Up,Rp,F,R2,Up,Rp,U2,R,Up,Rp}{}

\alg{Nb perm}{rp,Dp,F,r,Up,rp,Fp,D,r2,U,rp,Up,rp,F,r,Fp}{}

\alg{V perm}{R,Up,R,U,Rp,D,R,Dp,R,Up,D,R2p,U,R2,Dp,R2p}{}

\alg{Y perm}{F,R,Up,Rp,Up,R,U,Rp,Fp,R,U,Rp,Up,Rp,F,R,Fp}{}


\subsection{EPLL (Edge Permutation of the Last Layer)}
\section{EPLL (Edge Permutation of the Last Layer)}

\begin{multicols}{2}
	\alg{Z perm}{Mp,Up,(M2,Up),(M2,Up),Mp,U2,M2}{}

	\alg{H perm}{M2,U,M2,U2,M2,U,M2}{}

	\alg[{
		\draw[<-,thick] (0.5,1.5) -- (2.3,1.5);%
		\draw[->,thick] (0.4,1.4) -- (1.5,0.5);%
		\draw[->,thick] (1.6,0.5) -- (2.5,1.4);%
	}]{Ua perm}{M2,U,M,U2,Mp,U,M2}{}

	\alg[{
		\draw[->,thick] (0.5,1.5) -- (2.3,1.5);%
		\draw[->,thick] (2.5,1.5) -- (1.6,0.5);%
		\draw[<-,thick] (0.5,1.4) -- (1.5,0.5);%
	}]{Ub perm}{M2,Up,M,U2,Mp,Up,M2}{}

\end{multicols}




\end{document}
